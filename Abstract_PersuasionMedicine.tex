
 \documentclass[letterpaper,11pt]{article}
 \usepackage[margin=1in,footskip=0.4in]{geometry}
\usepackage{amsmath,amssymb,amsfonts,mathrsfs,accents} %important math packages
\usepackage{enumerate} % package for making different lists
\usepackage[T1]{fontenc} % Encoding of fonts
\usepackage[utf8]{inputenc} % Encoding of input text
\usepackage[kerning]{microtype} % Better looking text
\usepackage[babel]{csquotes} % Better looking quotes
\usepackage{booktabs} % Better looking tables
\usepackage[american]{babel} % Language control, for hyphenation etc
\usepackage{lmodern}
\usepackage{graphicx} 
\usepackage{float} 
\usepackage{secdot}
\usepackage{mathtools}
\usepackage{placeins}
\usepackage{lscape}
\usepackage[hyphenbreaks]{breakurl}
\usepackage[hyphens]{url}
\usepackage{scrextend}
\usepackage{parskip}
\setlength\parindent{0pt}
\usepackage{setspace}
\doublespacing

%\usepackage{subcaption}
\usepackage{chngcntr}
\usepackage{wasysym}
\usepackage[explicit]{titlesec}
\usepackage{multirow}
\usepackage{subcaption}
\usepackage{caption}
\usepackage{bbm}
\usepackage{booktabs}
\usepackage{tabularx}
\usepackage{docmute} % to skip the preamble in tabout files
\usepackage{epstopdf}
\usepackage{upgreek} %write curly epsilon
\usepackage{bbold} %write indicator 1
%\usepackage[capposition=top]{floatrow}

\usepackage{longtable}
\usepackage{lscape}
\renewcommand{\baselinestretch}{1}

\usepackage[round]{natbib}

%%%%%%%%%%%%%%%%%%%%%%%%%%%%%%%%%%%%%%%%%%%%%%%%%%%%%%%%%%%%%%%%%%%%%%%%%%%%%%%
% Here's where the document begins
%
\begin{document}

\Large{\textbf{Persuasion in Medicine}}

\small{\textit{With Marcella Alsan}}


%\end{flushright}
%\end{minipage}
 \bigskip

Abstract

\medskip

Take-up of preventive health care is often low, even when the care is available at low cost and widely recommended. How persuasive a recommendation is may depend on the identity of the recommender: Experts and socially proximate individuals may be more effective at updating others’ beliefs and influencing behavior - but for low-income individuals, expertise typically implies social distance. In this field experiment, we test how social proximity and expertise of the message sender influence attitudes, beliefs, valuations and take-up of seasonal flu vaccination among low-income men in the US. We first disseminate informercials about the flu vaccine, varying the race of the message sender, as well as their expertise (either doctor or lay person). We then distribute coupons for free flu shots and track their redemption.


\end{document}
