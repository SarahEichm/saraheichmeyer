
 \documentclass[letterpaper,11pt]{article}
 \usepackage[margin=1in,footskip=0.4in]{geometry}
\usepackage{amsmath,amssymb,amsfonts,mathrsfs,accents} %important math packages
\usepackage{enumerate} % package for making different lists
\usepackage[T1]{fontenc} % Encoding of fonts
\usepackage[utf8]{inputenc} % Encoding of input text
\usepackage[kerning]{microtype} % Better looking text
\usepackage[babel]{csquotes} % Better looking quotes
\usepackage{booktabs} % Better looking tables
\usepackage[american]{babel} % Language control, for hyphenation etc
\usepackage{lmodern}
\usepackage{graphicx} 
\usepackage{float} 
\usepackage{secdot}
\usepackage{mathtools}
\usepackage{placeins}
\usepackage{lscape}
\usepackage[hyphenbreaks]{breakurl}
\usepackage[hyphens]{url}
\usepackage{scrextend}
\usepackage{parskip}
\setlength\parindent{0pt}
\usepackage{setspace}
\doublespacing

%\usepackage{subcaption}
\usepackage{chngcntr}
\usepackage{wasysym}
\usepackage[explicit]{titlesec}
\usepackage{multirow}
\usepackage{subcaption}
\usepackage{caption}
\usepackage{bbm}
\usepackage{booktabs}
\usepackage{tabularx}
\usepackage{docmute} % to skip the preamble in tabout files
\usepackage{epstopdf}
\usepackage{upgreek} %write curly epsilon
\usepackage{bbold} %write indicator 1
%\usepackage[capposition=top]{floatrow}

\usepackage{longtable}
\usepackage{lscape}
\renewcommand{\baselinestretch}{1}

\usepackage[round]{natbib}

%%%%%%%%%%%%%%%%%%%%%%%%%%%%%%%%%%%%%%%%%%%%%%%%%%%%%%%%%%%%%%%%%%%%%%%%%%%%%%%
% Here's where the document begins
%
\begin{document}

\Large{\textbf{Can a Single Opioid Prescription Make a Difference? Evidence from Physician Prescribing Variation in Emergency Departments}}

\small{\textit{With Jonathan Zhang}}


%\end{flushright}
%\end{minipage}
 \bigskip

Abstract

\medskip

In the past two decades, death rates from opioids have seen a fivefold increase and opioid prescribing has emerged as a leading public health problem in the United States. Clinical guidelines leave many opioid prescribing decisions to physician judgement; we study the extent to which a single opioid prescription in an emergency department, for these marginal cases, can induce long-term dependence and impact health and economic outcomes of a patient. We tackle these questions by leveraging quasi-random assignment of patients to physicians, who vary in their propensity to prescribe opioids. We analyze the universe of electronic health record data for a particularly vulnerable population—veterans—and find that a single opioid prescription can have strong adverse effects on a veteran’s long-term outcomes. A single opioid prescription induces a 1.2 percentage point (pp) increase in the probability of long-term prescription opioid use, a 0.33pp increase in development of an opioid use disorder, and a 0.075pp increase in opioid overdose mortality. We find suggestive evidence of transition into illicit opioids. Moreover, in settings where the supply of legal prescription opioids is restricted, veterans are more likely to resort to illicit opioids, highlighting the complex interdependencies between legal and illicit sources of opioid supply.


\end{document}
